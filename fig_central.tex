\begin{myfig}
    {img_centralModel} % filename
    {0.9} % width
    {fig:central} % label
    {The central model of communication theory \citep[adapted from][p. 381 fig. 1]{shannon1948mathematicala}. 
    The \textbf{Source} is a process that generates strings from a lexicon, within which symbols have a certain probability of appearing. 
    The \textbf{Encoder} converts strings of the source lexicon into strings of the code lexicon. 
    The \textbf{Channel} is the medium through which code strings are transmitted. 
    During transmission, the code strings are subject to \textbf{Noise}, potentially changing their constituent symbols in a non-deterministic way. 
    The \textbf{Decoder} attempts to convert code strings back into strings of the original lexicon. 
    Finally, the \textbf{Target} is the reconstructed string. 
    The success of communication in this model is measured in terms of the probability of error; specifically, the probability that a symbol in the target string will differ from the corresponding symbol in the source string.
    } % caption
\end{myfig}