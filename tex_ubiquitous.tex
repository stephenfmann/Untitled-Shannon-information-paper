\section{Ubquitous information}\label{sec:ubiquitous}

\subsection{How scientists use information theory}\label{subsec:scientists}

The tools of information theory are statistical tools.
Chief among them is mutual information, typically interpreted as a measure of the strength of correlation between two variables.
Mutual information has been employed in many ways across diverse sciences, including:

\begin{itemize}
    \item Behavioural ecology, to measure the correlation between the honey bee waggle dance and the location of food sources \citep{haldane1954statistical}
    \item Molecular biology, to measure the correlation between inputs and outputs of a quorum-sensing bacterium \citep{mehta2009information}
    \item Evolutionary biology, to measure the growth rate of an organism conditioning its behaviour on an informational cue \citep{donaldson-matasci2010fitness}
    \item Cosmology, to measure the correlation between galaxies' internal morphology and their local environments \citep{pandey2017how}
    \item Linguistics, to measure the co-occurrence of words in a corpus \citep[$\S$4]{hunston2002corpora}
\end{itemize}

