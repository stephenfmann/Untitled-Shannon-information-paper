\section{Introduction: blocking the path to scepticism}\label{sec:intro}

Communication theory measures the costs and benefits of representation, and describes judicious representational trade-offs. 
There is a popular idea that the features of representations communication theory deals with are strictly distinct from representational content.
For example, this distinction is sometimes assimilated to quantitative and qualitative features of representations.
Communication theory deals with transmission rates and capacity -- a quantitative concept, often called Shannon information -- whereas what typically motivates philosophical theories is representational content -- a qualitative concept, sometimes called semantic information.
While cognitive science nowadays uses tools from communication theory (especially measures of correlation like mutual information) to understand quantitative aspects of cognitive representations, it is common to suggest that theories of representational content will be in some way orthogonal or unrelated to communication theory.
This motivates philosophical theorising without reference to the tools of communication theory, and appears to justify claims that those tools are inappropriate resources to draw on when developing a theory of content.

This paper rejects the orthodox separation of communication theory from theories of representational content, by undercutting its two primary justifications.
I'll refer to my opponents as "sceptics", by which is meant scepticism about the relevance of communication theory for philosophical theories of representational content.
The remainder of this introduction outlines the two main ways this scepticism is justified, by way of describing the structure of the rest of the paper.

First, scholars often appeal to a warning given by Claude Shannon, the founder of communication theory, that the term `information' as applied in the theory should be sharply distinguished from the colloquial term `meaning'.
While Shannon certainly did make this claim, philosophers have come to interpret him as saying something stronger: that signals in communication-theoretic models need not mean anything.
Properly interpreted, Shannon's warning in no way justifies that claim, and therefore does not justify currently prevalent scepticism.
It turns out that Shannon made use of a pretheoretic notion of content in describing signals, and that contemporary communication theorists continue to use terms like `represent' and `describe' in these contexts.
I argue that teleosemantics captures their usage because it treats signals in communication-theoretic models as representations.

Second, many authors note that certain mathematical tools developed by Shannon can be applied in contexts far removed from signalling systems.
They conclude that these tools are insufficient on their own to capture what is philosophically interesting about representation.
I argue that although it is true that measures like entropy and mutual information cannot distinguish the signal-signified relationship from other correlational relationships, sceptics undersell the resources available to communication theory.
The theory involves models that define signals, and theorems that describe the costs and benefits of transmitting and responding to signals.
These theoretical results apply specifically to signalling systems, not just any correlational relationship, and can therefore describe more interesting properties of representations than just quantities of mutual information.

Communication theory ought to be expanded to describe representations in cognitive systems, and ought to be embedded in naturalist approaches to content.
I emphasise again that the use of communication-theoretic tools in cognitive science is commonplace, and sceptics are not generally complaining about that.
Rather, sceptical assertions imply that all of the theoretical and practical work of communication theory could go on without any reference to the contents of representations, and without any consideration of the way in which representations acquire their contents.
It is this that I take to be orthodoxy, and it is the justification of this kind of claim that I argue against here.

