\section{Introduction: blocking the path to scepticism}\label{sec:intro}

Communication theory measures the costs and benefits of representation, and describes judicious representational trade-offs. 
Recent work by \citet{martinez2019representations} uses communication theory to answer philosophical questions about cognitive representations.
However, the idea that communication theory can be applied in this way is subject to long-standing and prominent scepticism about the relevance of mathematical formalism for naturalist theories of representation.
The very possibility of Mart\'{i}nez's project is in question.

In this paper I describe and undercut two popular justifications for that scepticism.
We are left with good reason to endorse and extend Mart\'{i}nez's positive claims.
The remainder of this introduction outlines those two lines of scepticism, by way of describing the structure of the rest of the paper.

First, scholars often appeal to a warning given by Claude Shannon, the founder of communication theory, that the term `information' as applied in the theory should be sharply distinguished from the colloquial term `meaning'.
While Shannon certainly did make this claim, philosophers have come to interpret him as saying something stronger: that signals in communication-theoretic models need not mean anything.
Properly interpreted, Shannon's warning in no way justifies that claim, and therefore does not justify currently prevalent scepticism.
It turns out that Shannon made use of a pretheoretic notion of content in describing signals, and that contemporary communication theorists continue to use terms like `represent' and `describe' in these contexts.
I argue that teleosemantics captures their usage because it treats signals in communication-theoretic models as representations.

Second, many authors note that certain mathematical tools developed by Shannon can be applied in contexts far removed from signalling systems.
They conclude that these tools are insufficient on their own to capture what is philosophically interesting about representation.
I argue that although it is true that measures like entropy and mutual information cannot distinguish the signal-signified relationship from other correlational relationships, sceptics undersell the resources available to communication theory.
The theory involves models that define signals, and theorems that describe the costs and benefits of transmitting and responding to signals.
These theoretical results apply specifically to signalling systems, not just any correlational relationship, and can therefore describe more interesting properties of representations than just quantities of mutual information.

Traditional objections to Mart\'{i}nez's account are unwarranted.
Communication theory ought to be expanded to describe representations in cognitive systems, and ought to be embedded in naturalist approaches to content.
