\section{Introduction: blocking the path to scepticism}\label{sec:intro}

Communication theory measures the costs and benefits of representation, and describes judicious representational trade-offs. 
Recent work by \citet{martinez2019deception,martinez2019representations} uses communication theory to answer long-standing philosophical questions about cognitive representations.
However, the assumption that communication theory is apt to be applied in this way continues to be challenged by prominent scepticism about the relevance of mathematical formalism for naturalist theories of representation.
This leads to pessimism about the very possibility of Mart\'{i}nez's project.

In this paper I describe and undercut two popular justifications for that scepticism.
We are left with good reason to endorse and extend Mart\'{i}nez's positive claims.
The remainder of this introduction outlines those two lines of scepticism, by way of describing the structure of the rest of the paper.

First, scholars often appeal to a warning given by Claude Shannon, the founder of communication theory, that the term `information' as applied in the theory should be sharply distinguished from the colloquial term `meaning'.
While Shannon certainly did make this claim, philosophers have come to interpret him as saying something much stronger: that signals in communication-theoretic models need not mean anything.
Properly interpreted, Shannon's warning in no way justifies that claim, and therefore does not justify currently prevalent scepticism.
Contrary to the sceptical conclusion, Shannon made use of a pretheoretic notion of `meaning' in describing signals, and contemporary communication theorists continue to use terms like `represent' and `describe' in these contexts.
I argue that teleosemantics captures their usage because it treats signals in communication theory models as representations.

Second, many authors note that the mathematical tools developed by Shannon can be applied in contexts far removed from signalling and representational systems.
They conclude that these tools are insufficient on their own to capture what is philosophically interesting about representation.
I argue that this line of thought is a consequence of a pair of mistakes.
By focusing solely on measures like mutual information, sceptics undersell the resources available to communication theory when capturing interesting features of representations.
And by conflating communication theory and information theory, sceptics ignore that the interpretation of mathematical measures is relevant to the naturalist project when those measures are applied to representational systems.

We are left with no reason to object to Mart\'{i}nez's positive account.
Communication theory ought to be expanded to describe representations in cognitive systems, and ought to be embedded in naturalist approaches to content.

% We are left with a view of communication theory that recommends fully integrating it into a naturalist theory of representation. 
% The most suitable existing account is teleosemantics, which should be viewed as a generalisation of communication theory beyond the paradigm of binary digits and electrical signals upon which it was founded.  