%%%%%%%%%%%%%
%%%%%%%%%%%%%
%%%%%%%%%%%%%
%%%%%%%%%%%%%
\section{Extending communication theory}\label{sec:extending}

\subsection{Extending communication theory to cognitive science}

I hope to have dispelled some of the scepticism that leads philosophers to disregard communication theory as a mathematical treatment that deals only with `Shannon information'.
However, the overall point of this paper is more specific: I am trying to remove obstacles to the application of communication theory in cognitive science.
A few words are therefore in order addressing that application.

What do we actually gain by studying the brain in the light of an engineering discipline?

\begin{itemize}
    \item Mart\'{i}nez's basic claims
    \item Examples of comm theory concepts in cog sci proper e.g. Sims on rate-distortion theory
\end{itemize}

%%%%%%%%%%%
%%%%%%%%%%%
%%%%%%%%%%%
%%%%%%%%%%%

\subsection{Extending communication theory to a general theory of teleosemantics}

\begin{itemize}
    \item The general lesson: communication theory is about the cost of representation, and how those costs trade off against the benefits of representation. There is no reason why such a theory would be limited to binary symbols in digital channels. Its claims are universal.
    \item Future: what we really need is a general theory of function.
\end{itemize}