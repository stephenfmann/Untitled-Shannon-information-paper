\begin{myfig}
    {img_teleoBasic} % filename
    {0.9} % width
    {fig:teleo} % label
    {\textbf{The basic teleosemantic model}. 
    A \textbf{Sender} and \textbf{Receiver} have a joint goal they must achieve. 
    This is modelled as the requirement of setting a \textbf{Success?} variable to a certain value. 
    The receiver can exercise some control by \textbf{Act}ing, but a \textbf{Distal State} also has causal influence over the target. 
    The sender can produce a \textbf{Signal} that bears an exploitable relation to the distal state.
    When the receiver can condition its behaviour on the signal and achieve greater success than otherwise, teleosemantics claims that explaining this improvement requires appealing to a relation between signal and distal state.
    This relation is semantic content.
    % For this to work the sender and receiver must operate under the same assumptions about how signals are related to distal states. 
    % Given these assumptions -- these mapping rules -- a given signal will have a certain external state as its semantic content. 
    Finally, how the sender actually produces such a signal is usually by conditioning on one or more upstream \textbf{Proximate States}.
    In special cases, the Proximate and Distal states are identical (see figures \ref{fig:central} and \ref{fig:sr}).
    Adapted from \citet[fig. 6.3, p. 78]{millikan2004varieties}.
    } % caption
\end{myfig}