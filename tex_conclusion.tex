%%%%%%%%%%%%%
%%%%%%%%%%%%%
%%%%%%%%%%%%%
%%%%%%%%%%%%%
\section{\sloppy Concluding remarks: communication theory for cognitive science}\label{sec:conclusion}

I hope to have dispelled some of the scepticism that leads philosophers to disregard communication theory as a mathematical treatment that deals only with `Shannon information'.
However, the overall point of this paper is more specific: I am trying to remove obstacles to the application of communication theory in cognitive science.
A few words are therefore in order addressing that application.

What do we actually gain by studying the brain in the light of an engineering discipline?
Engineering constraints sometimes prove to be apt descriptions of evolved objects because natural selection sometimes produces objects that achieve goals while being subject to performance constraints.
Communication theory is ultimately a theory about the cost of representation, and how those costs trade off against the benefits of representation.
There is no reason why such a theory should be limited to binary symbols in digital channels.
Its concepts ought to be universally valid.
If the brain represents at all, its ability to improve the accuracy of its representations must be subject to performance constraints.
Communication theory is the theoretical framework within which that trade-off is defined and optimally achieved.
[LEAD INTO SIMS EXAMPLE]

\begin{itemize}
    \item Examples of comm theory concepts in cog sci proper e.g. Sims on rate-distortion theory
    \item Mart\'{i}nez's basic claims
\end{itemize}