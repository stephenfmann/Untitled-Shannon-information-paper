%%%%%%%%%%%%%
%%%%%%%%%%%%%
%%%%%%%%%%%%%
%%%%%%%%%%%%%
\section{Concluding remarks: communication theory for cognitive science}\label{sec:conclusion}

I hope to have dispelled some of the scepticism that leads philosophers to disregard communication theory as a mathematical treatment that deals only with `Shannon information'.
However, the overall point of this paper is more specific: I am trying to remove obstacles to the application of communication theory in cognitive science.
A few words are therefore in order addressing that application.

What do we actually gain by studying the brain in the light of an engineering discipline?
Engineering constraints sometimes prove to be apt descriptions of evolved objects because natural selection sometimes produces objects that achieve goals while being subject to performance constraints.
Communication theory is ultimately a theory about the cost of representation -- the effort required to send and receive signals -- and how those costs trade off against the benefits of representation -- the performance improvement a receiver enjoys by conditioning its behaviour on the signal.
There is no reason why such a theory should be limited to binary symbols in digital channels.
Its concepts ought to be universally valid.
If the brain represents at all, its ability to improve the accuracy of its representations must be subject to performance constraints.
Communication theory is the theoretical framework within which that trade-off is defined and optimally achieved.

Can cognitive science apply the principles of communication theory to understand the brain?
It already does.
\citet{sims2016ratedistortion} describes human performance in perceptual tasks in terms of a trade-off between the capacity of perceptual information transmission and the cost of perceptual error.
This is the same trade-off at the heart of the third theorem of communication theory, which launched an important subdiscipline called \textbf{rate-distortion theory}.
Perceptual information rate can be increased, lowering the chance of perceptual error, by expending more metabolic resources.
Whether it is worth investing resources to increase perceptual accuracy depends on the costs of inaccuracy.
Sims derives a cost function assumed to be operative in human subjects from experiments that push the limits of their perceptual capacity.
The pattern of errors made by subjects reveals a consistent cost function used across experimental conditions \citep[188]{sims2016ratedistortion}.
Sims situates the role of the theory at the computational level of explanation:

\begin{myquote}
Rate-distortion theory combines the central elements of both information theory and decision theory, and is uniquely situated for explaining biological computation as a principled, but capacity-limited system. As a computational-level theory, the goal is not to contradict explanations formulated at the neural or algorithmic level, but rather provide an explanation for the `why' of behavior, and provide inspiration for the development of mechanistic theories.
\par\hspace*{\fill}\citet[193]{sims2016ratedistortion}
\end{myquote}

\citet{martinez2019representations} explores the role of representation in cognitive science at the computational level too.
He argues that mainstream accounts of representation -- in particular, robust-tracking accounts due to \citet{sterelny2003thought,burge2010origins}, and reference-magnet accounts due to \citet{ryder2004sinbad,lewis1984putnam} -- should be understood in terms of rate-distortion theory.
In brief, robust tracking is a cognitive capacity enabling a creature to use multiple perceptual sources of information to infer the presence of some success-relevant distal state (say, the presence of a predator); Mart\'{i}nez argues that the informational structure of the world -- how the predator's presence generates those varied perceptual inputs -- defines a characteristic rate-distortion relationship within which there is a \textit{sweet spot}.
Representations result from compressing those percepts in a way that exploits this sweet spot.
Reference-magnetism translates into communication-theoretic language in a similar way \citep[1223]{martinez2019representations}.
Interested readers are encouraged to read the paper; here my goal has been to forestall popular lines of dissent to the very possibility of Mart\'{i}nez's project.

