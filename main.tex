\documentclass[12pt]{article}


			%%%%%%%%%%%%%%%%%%%%%%%%%%%%%%
			%%%%%%%%%%%%%%%%%%%%%%%%%%%%%%
			%%% Generic LaTeX preamble %%%
			%%%%%%%%%%%%%%%%%%%%%%%%%%%%%%
			%%%%%%%%%%%%%%%%%%%%%%%%%%%%%%


			%%%%%%%%%%%%%%%%%%%%%%%%%%%%%%
			%%%%%%%%%%%%%%%%%%%%%%%%%%%%%%
			%%%% (1) TITLE AND AUTHOR %%%%
			%%%%%%%%%%%%%%%%%%%%%%%%%%%%%%
			%%%%%%%%%%%%%%%%%%%%%%%%%%%%%%

\title{The relevance of communication theory for theories of representation}
% \author{}
\date{\vspace{-10ex}}

			%%%%%%%%%%%%%%%%%%%%%%%%%%%%%%
			%%%%%%%%%%%%%%%%%%%%%%%%%%%%%%
			%%%%%% (2) FORMATTING %%%%%%%%
			%%%%%%%%%%%%%%%%%%%%%%%%%%%%%%
			%%%%%%%%%%%%%%%%%%%%%%%%%%%%%%

%% Font
\usepackage{fontenc}
\usepackage{graphicx}
\usepackage{kpfonts,baskervald}
\graphicspath{{img/}}

%% Margins and line spacing
\usepackage[
	%showframe, % for debugging
	a4paper,
	total={6in, 8in}
]{geometry}
\renewcommand{\baselinestretch}{1.5} % line spread
\setlength{\voffset}{-0.75in} % top of page
\setlength{\textheight}{700pt}
\setlength{\marginparwidth}{0pt}

\usepackage{mathptmx}
\usepackage{mathtools} % use \coloneqq for definitions
\usepackage{amsmath}
\usepackage{hyperref}

%% Table stuff
\usepackage{makecell} % line breaks within cells
\renewcommand{\cellalign}{cl} % makecell: text inside \makecell{} is vertically [c]entered, horizontally [l]eft-aligned


			%%%%%%%%%%%%%%%%%%%%%%%%%%%%%%
			%%%%%%%%%%%%%%%%%%%%%%%%%%%%%%
			%%%%% (3) BIBLIOGRAPHY %%%%%%%
			%%%%%%%%%%%%%%%%%%%%%%%%%%%%%%
			%%%%%%%%%%%%%%%%%%%%%%%%%%%%%%

\usepackage{doi}
\usepackage[style=authoryear-comp,
		backend=bibtex,
		isbn=false,
		natbib=true,
    uniquelist=false, % multiple authors need not be identical before et al is used
    maxcitenames=2] % more than two authors and citation switches to et al
		{biblatex}

\addbibresource{library.bib}

\renewbibmacro{in:}{}

%% Don't print both DOI and URL
\renewbibmacro*{doi+eprint+url}{%
  \iftoggle{bbx:doi}
    {\iffieldundef{url}{\printfield{doi}}{}}
    {}%
  \newunit\newblock
  \iftoggle{bbx:eprint}
    {\usebibmacro{eprint}}
    {}%
  \newunit\newblock
  \iftoggle{bbx:url}
    {\usebibmacro{url+urldate}}
    {}}

%% Customise bibliography fields
\AtEveryBibitem{
    \clearlist{language}	% Nobody cares
    \clearfield{urldate}	% Needless `visited on' field
    \clearfield{urlyear}	% Needless `visited on' field
    \clearfield{month}	% In bracketed date 
    \clearfield{timestamp}
    \clearfield{note}		% Not needed
    \iffieldequals{doi}{*}{
    	\clearfield{url}
    }
}

%% Custom 'Link' option (instead of printing entire URL or DOI)
\DeclareFieldFormat{url}{\href{#1}{\underline{Link}}}
\DeclareFieldFormat{doi}{\href{https://doi.org/#1}{\underline{Link}}}



			%%%%%%%%%%%%%%%%%%%%%%%%%%%%%%
			%%%%%%%%%%%%%%%%%%%%%%%%%%%%%%
			%%%%% (4) ENVIRONMENTS %%%%%%%
			%%%%%%%%%%%%%%%%%%%%%%%%%%%%%%
			%%%%%%%%%%%%%%%%%%%%%%%%%%%%%%

%% Figure environment
\newenvironment{myfig}[4]
% title, width (as proportion of textwidth), label, caption
{ % Inside these braces is what comes BEFORE the text inside the myfig environment.
  \begin{figure}
  \begin{center}
  \includegraphics[width=#2\textwidth]{#1}
  \caption{\label{#3} #4}
}
{ % Inside these braces is what comes AFTER the text inside the myfig environment.
  \end{center}
  \end{figure}
}

			%%%%%%%%%%%%%%%%%%%%%%%%%%%%%%
			%%%%%%%%%%%%%%%%%%%%%%%%%%%%%%
			%%%% (5) CUSTOM COMMANDS %%%%%
			%%%%%%%%%%%%%%%%%%%%%%%%%%%%%%
			%%%%%%%%%%%%%%%%%%%%%%%%%%%%%%
			
%% Quote environment
\newenvironment{myquote}
{ % Inside these braces is what comes BEFORE the text inside the myquote environment.
  \begingroup
  \vspace{0.5em}
  \begin{quote}
  %\onehalfspacing
}
{ % Inside these braces is what comes AFTER the text inside the myquote environment.
  \end{quote}
  \endgroup
}

%%% Names of claims
\newcommand{\ami}{{\sc Mutual Information is Agnostic}}
\newcommand{\ait}{{\sc Information Theory is Agnostic}}
\newcommand{\act}{{\sc Communication Theory is Agnostic}}
\newcommand{\cia}{{\sc Channel Capacity is Agnostic}}

\begin{document}

\maketitle

%%%
%%%
%%%

%TC:ignore
\begin{abstract}
\noindent Prominent views about representation share a premise: that mathematical communication theory is blind to representational content. Here I challenge that premise by rejecting two common misconceptions: that Claude Shannon said that the meanings of signals are irrelevant for communication theory (he didn't and they aren't), and that since correlational measures can't distinguish representations from natural signs, communication theory can't distinguish them either (the premise is true but the conclusion is false; no valid argument can link them).
\end{abstract}
%TC:endignore

%%%
%%%
%%%
\section{Introduction: blocking the path to scepticism}\label{sec:intro}

Communication theory measures the costs and benefits of representation, and describes judicious representational trade-offs. 
There is a popular idea that the aspects of representation communication theory deals with are strictly distinct from representational content:

\begin{myquote}
[Communication theory] ignores questions having to do with the \emph{content} of signals, what \emph{specific information} they carry, in order to describe \emph{how much} information they carry.
\par\hspace*{\fill}\citet[41]{dretske1981knowledge}, emphasis original
\end{myquote}

\begin{myquote}
Shannon-Weaver theory measures the \emph{capacity} of information-transmission and information-storage vehicles, but is mute about the \emph{contents} of those channels and vehicles, which will be the topic of the still-to-be-formulated theory of semantic information.
\par\hspace*{\fill}\citet[344]{dennett1983intentional}, emphasis original
\end{myquote}

\begin{myquote}
Shannon information does not capture, nor is it intended to capture, the semantic content, or meaning, of signals.
\par\hspace*{\fill}\citet[21]{piccinini2011information}
\end{myquote}

\begin{myquote}
Shannon’s theory, taken in itself, is purely quantitative: it ignores any issue related to informational content.
\par\hspace*{\fill}\citet[1989]{lombardi2015shannon}
\end{myquote}

\begin{myquote}
Shannon offers no analysis of the relation in virtue of which a sign carries information \textit{about} a state of affairs (his interest was in other issues).
\par\hspace*{\fill}\citet[7]{neander2017mark}, emphasis original
\end{myquote}

\noindent The problem with these claims -- the tension that this paper will attempt to resolve -- is that they appear to be contradicted by the words of communication theorists themselves:

\begin{myquote}
[Efficiency is achieved] in telegraphy by using the shortest channel symbol, a dot, for the most common English letter E; while the infrequent letters, Q, X, Z are \textit{represented by} longer sequences of dots and dashes. This idea is carried still further in certain commercial codes where common words and phrases are \textit{represented by} four- or five-letter code groups with a considerable saving in average time.
\par\hspace*{\fill}\citet[385]{shannon1948mathematicalc}, emphasis added
\end{myquote}

\begin{myquote}
Thus the messages of high probability are \textit{represented by} short codes and those of low probability by long codes.
\par\hspace*{\fill}\citet[402]{shannon1948mathematicalc}, emphasis added
\end{myquote}

\begin{myquote}
[The source coding theorem] provides another justification for the definition of entropy rate -- it is the expected number of bits per symbol required to \textit{describe} the process.
\par\hspace*{\fill}\citet[115]{cover2006elements}, emphasis added
\end{myquote}

\begin{myquote}
We can design source codes for the most efficient \textit{representation} of the data. [...] The common representation for all kinds of data uses a binary alphabet. Most modern communication systems are digital, and data are reduced to a binary representation for transmission over the common channel.
\par\hspace*{\fill}\citet[218]{cover2006elements}, emphasis added
\end{myquote}

\begin{myquote}
We now discuss the information content of a source by considering how many bits are needed to \textit{describe} the outcome of an experiment.
\par\hspace*{\fill}\citet[73]{mackay2003information}, emphasis added
\end{myquote}

\noindent Philosophers say that communication theory ignores the content of signals, but communication theorists habitually refer to signal content, using verbs such as `represent' and `describe'.
As we will see, these locutions play a role in interpreting theorems and justifying mathematical concepts.
What's going on?

This paper resolves the puzzle by arguing that philosophers have misunderstood communication theory.
This resolution should be taken seriously because it respects the justifications philosophers themselves give for their claims.
Philosophers typically appeal to two lines of justification for claims of the sort above.
I argue that both lines of justification are unsupportable given a proper understanding of communication theory.

First, scholars often appeal to a warning given by Claude Shannon, the founder of communication theory, that the term `information' as applied in the theory should be sharply distinguished from the colloquial term `meaning'.
We'll see in section \ref{sec:warning} that Shannon did make this claim.
However, philosophers have come to interpret him as saying something stronger: that signals in communication-theoretic models need not mean anything.
Properly interpreted, Shannon's warning does not justify that sceptical claim.
In the above quotes, Shannon makes use of a pretheoretic notion of content in discussing signals.
Contemporary communication theorists continue to use terms like `represent' and `describe' in these contexts.

Second, many authors note that certain mathematical tools developed by Shannon can be applied in contexts far removed from signalling systems.
They conclude that these tools are insufficient on their own to capture representational content.
I argue in section \ref{sec:agnostic} that although it is true that mathematical measures like mutual information cannot distinguish the signal-signified relationship from other correlational relationships, sceptics undersell the resources available to communication theory.
The theory involves models that define signals, and theorems that describe the costs and benefits of transmitting and responding to signals.
These theoretical results apply specifically to signalling systems, not just any correlational relationship, and can therefore describe more interesting properties of signals than just quantities of mutual information.

Before all that, in section \ref{sec:positive} I offer a positive account of representational content.
It is a teleosemantic account that defines content in terms of function.
I argue that the central model of communication theory is a special case of the basic teleosemantic model, and that teleosemantics therefore endorses the content-talk of communication theorists.
My refutations in subsequent sections do not depend strictly on the positive view, but I've found it helps to have an account of content on the table.



%%%
%%%
%%%
\section{The positive view}\label{sec:positive}

Before arguing against the sceptical claim, it can be useful to have a positive account on the table.
In this section I describe Millikan's teleosemantics and show that it attributes content to communication-theoretic signals.
These attributions agree with the contents communication theorists themselves ascribe to their signals.

\subsection{Teleosemantics}
I adopt a liberal teleosemantic theory of representation.
This view attributes representational content to a wide range of artificial and natural systems.
Anything that satisfies the model depicted in figure \ref{fig:teleo} is a representation system, and anything that plays the signal role in that system is a representation.
I'll briefly describe teleosemantics and how it motivates the claim that representational content is essential to communication theory.

\begin{myfig}
    {img_teleoBasic} % filename
    {0.9} % width
    {fig:teleo} % label
    {The basic teleosemantic model. 
    A \textbf{Sender} and \textbf{Receiver} have a joint goal they must achieve. 
    This is modelled as the requirement of setting a \textbf{Target} variable to a certain value. 
    The receiver can exercise some control, but a \textbf{Distal State} also has causal influence over the target. 
    The sender can produce a \textbf{Signal} that bears an exploitable relation to the distal state.
    When the receiver can condition its behaviour on the signal and achieve greater success than otherwise, teleosemantics claims that explaining this improvement requires appealing to a relation between signal and distal state.
    This relation is semantic content.
    % For this to work the sender and receiver must operate under the same assumptions about how signals are related to distal states. 
    % Given these assumptions -- these mapping rules -- a given signal will have a certain external state as its semantic content. 
    Finally, how the sender actually produces such a signal is usually by conditioning on one or more upstream \textbf{Proximate States}.
    In special cases, the Proximate and Distal states are identical (see figures \ref{fig:central} and \ref{fig:sr}).
    Adapted from \citet[fig. 6.3, p. 78]{millikan2004varieties}.
    } % caption
\end{myfig}

Some artificial and natural devices have the structure and dispositions they do because of causal effects they are supposed to bring about.
For artificial devices such as claw hammers, pumps, and communication systems, this link between a device's features (its structure and dispositions) and its proper effects (those it is supposed to bring about) is established by intentional design: a human engineer wanted certain effects to occur, and created the device so as to reliably produce those effects.
For natural devices such as claws, hearts, and neurons, the link between features and proper effects is established by natural selection (operating over phylogenetic lineages) and perhaps ontogenic selection (operating within an organism during development).
Whichever way the link is established, a device that was designed or selected for producing certain causal effects has a proper function, and its features can be at least partly explained by reference to its function \citep[$\S\S$1-2]{millikan1984language}, \citep[$\S$2]{millikan1993white}.

When two devices -- call them sender and receiver -- are designed or selected to assist each other in producing an outcome (as in figure \ref{fig:teleo}), an intermediary may enable coordination between them by bearing a relation to a distal state.
The distal state causally influences the outcome sender and receiver are trying to bring about, so the receiver could be more successful by conditioning its activity on the distal state.
When the receiver uses the intermediary as a proxy and enjoys greater success as a result, its improvement must be explained by reference to a relation between the intermediary and the distal state.
Teleosemantics says that when these conditions are met, the intermediary is a representation, and the distal state is its content.


%%%
%%%
%%%
\subsection{Teleosemantics attributes content to engineered signals}

Communication theory concerns devices that have the function of transmitting and responding to signals.
In this sense it deals with the very same kinds of system as teleosemantics.
The difference between the two theories lies in the questions they attempt to answer: teleosemantics tries to define representations and state their contents, while communication theory tries to devise methods for improving the performance of sender and receiver functions.
These questions are typically treated as orthogonal to each other.
In section \ref{subsec:signalContent} I'll argue this is false: improving performance of a signalling system ineliminably involves considerations of signal content.
Here I introduce communication theory.

The central engineering model of communication (figure \ref{fig:central}) construes the goal of communication as reproducing, at a target location, a symbol string produced at a spatiotemporally distant source.
The goal is achieved by \textbf{encoding} the source string, which means converting it into a sequence of physical events (typically electrical pulses) that can be transmitted as a signal across a channel.
At the far end of the channel the signal is decoded, producing a target string.
Communication is deemed successful when the target string is sufficiently similar to the source string.
(How similar the two strings need to be to count as `sufficiently similar' will differ depending on the context.)

\begin{myfig}
    {img_centralModel} % filename
    {0.9} % width
    {fig:central} % label
    {\textbf{The central model of communication theory} \citep[adapted from][p. 381 fig. 1]{shannon1948mathematicalc}. 
    The \textbf{Source} is a process that generates strings from a lexicon, within which symbols have a certain probability of appearing. 
    The \textbf{Encoder} converts strings of the source lexicon into strings of the code lexicon. 
    The \textbf{Channel} is the medium through which code strings are transmitted. 
    During transmission, the code strings are subject to \textbf{Noise}, potentially changing their constituent symbols in a non-deterministic way. 
    The \textbf{Decoder} attempts to convert code strings back into strings of the original lexicon. 
    Finally, the \textbf{Target} is the reconstructed string. 
    The success of communication in this model is measured in terms of the probability of error; specifically, the probability that a symbol in the target string will differ from the corresponding symbol in the source string.
    } % caption
\end{myfig}


Setting noise aside, the central model is a special case of the basic teleosemantic model.
% Teleosemantics therefore attributes content to signals in the communication-theoretic model, and asserts that this content explains the success of engineered communication systems.
% Drawing on the quotes in the introduction, I argue in section \ref{sec:warning} that communication theorists endorse the view of signals as contentful representations.
% First let's investigate the relationship between the two models.
%
% How does the central model map onto the teleosemantic model? 
In the teleosemantic model, a distal state sits causally upstream of a target that the receiver has causal influence over.
The signal has this distal state as its content, because the receiver can achieve greater success by conditioning its act on the signal.
In the central model, source strings play the role of both distal and proximate states (though more recent models in communication theory distinguish them, e.g. \citet{berger1996ceo}).
Applying the basic teleosemantic model to the central model, the encoder is a sender and the decoder is a receiver.
Sender and receiver share a proper function as a consequence of design: to reconstruct the source string.
% \footnote{To clarify: the sender's most immediate proper function is to produce a signal that bears a certain relation to the source string. But it has this function in part because of its slightly less immediate function, shared with the receiver, of reconstructing the source string at the target.}
They achieve this goal by means of a signal whose form is determined by a code.
Source strings are encoded into signals, and signals are decoded into target strings.
Since the encoding defines the relation the signal must bear to the source string in order for the receiver to be successful, teleosemantics says the code gives the content for each signal.
The content of a signal is the source string it encodes.
% This accounts for the communication theorists' attributions of representational content to signals.

There are a couple of differences between figures \ref{fig:teleo} and \ref{fig:central} that mean the central model is not quite a special case of the teleosemantic model as depicted.
The teleosemantic model does not include noise, but adding a noise variable would not affect the definition of semantic content.
The `Success?' variable, and the causal link to it from the relevant distal state, is omitted from the central model; nonetheless, in the central model the receiver's act together with a distal state determines the joint success of the signalling partnership via an error measure.
An example will make this clear.

Consider a system that transmits outcomes of coin tosses $\{H,T\}$.
Each time the coin is tossed at the source, the task of the decoder is to produce the appropriate symbol $H$ or $T$ matching the result of the toss.
Suppose the sender transmits signals according to the code $H\rightarrow1,\ T\rightarrow0$, and the result of three coin tosses is $H, T, H$.
Then, assuming no noise, when the decoder receives the signal $101$ it correctly produces the sequence $HTH$.
Communication is successful because the reproduced string matches the original sequence of outcomes.
The signal $101$ represents the source sequence $HTH$, and that is how the decoder successfully reproduces it.\footnote{The claim is not that the content of the signal explains how the receiver produces the string `HTH' at the target. The claim is that the content of the signal explains how the receiver successfully reproduces the source string at the target. In general, teleosemantics claims not that content explains behaviour, but that content explains success. Figure \ref{fig:teleo} makes that explicit by distinguishing the Act variable from the Success? variable.}

Nothing has yet been said about mathematical measures like mutual information or entropy.
Neither teleosemanticists nor communication theorists are claiming that content is identified with, or constructed from, correlational relationships.
Rather, it is the encoding scheme that determines the content of a signal.
The reason the encoding scheme determines content is that it is the encoding scheme that enables the receiver to enjoy improved success in its goal of reproducing the source string.
Without a shared code, the receiver would be guessing blindly; thanks to the code, the receiver enjoys increased success; therefore, the code determines the content of the signal, which is the source string.

So far, I have briefly surveyed positive reasons for thinking communication theory involves considerations of representational content.
This conflicts with the orthodox, sceptical view, which takes communication theory to be mute or ignorant about content.
If we stopped here, we would be left with a puzzle: how to reconcile orthodox scepticism with the practice of communication theory?
Instead of fleshing out the positive view, the rest of the paper aims to undercut the scepticism.
The next two sections describe and refute two popular sceptical arguments about the relationship between communication theory and representational content.


%%%%%%%%%%%%%%%%%%%%%%%%%%%%%%%%%%%%%%%%%
%%%%%%%%%%%%%%%%%%%%%%%%%%%%%%%%%%%%%%%%%
%%%%%%%%% Shannon's Warning %%%%%%%%%
%%%%%%%%%%%%%%%%%%%%%%%%%%%%%%%%%%%%%%%%%
%%%%%%%%%%%%%%%%%%%%%%%%%%%%%%%%%%%%%%%%%

\section{First sceptical argument: Shannon's Warning}\label{sec:warning}

The first route to scepticism about the relevance of communication theory for theories of representational content begins with the claim that Claude Shannon, the founder of communication theory, warned that his theory had nothing to do with meaning.
In this section I argue that what Shannon actually said has been misconstrued by philosophers.
He was not talking about signals, but about sources, a different aspect of his model.
When Shannon did turn to signals, he called them representations and explicitly referred to them as contentful.
Contemporary communication theorists endorse this pretheoretic attribution of content, and teleosemantics accounts for it.


%%%%%%%%%%%%%%%%%%%%%%%%%%%
%%%%%%%%%%%%%%%%%%%%%%%%%%%
%%%%%%%%%%%%%%%%%%%%%%%%%%%

\subsection{In the central model, efficient signalling depends on source probabilities}
The heavy cryptographic and communicative demands of the Second World War led mathematicians and engineers, spearheaded by Claude Shannon, to develop the discipline known today as communication theory.
Published soon after the war's conclusion, the insights of Shannon's foundational text \parencite*{shannon1948mathematicalc} are predicated on the central model (figure \ref{fig:central}).
As we saw above, the goal of communication in this model is to reproduce a source string at a target location.
What is important is that communication is pitched as a \textit{syntactic} enterprise: it is the transmission and reconstruction of symbol strings, not the conveyance of their meanings, that is in question.

To give an example, suppose the lexicon from which the source string is constructed is the Latin script $\{A,B,C...\}$ plus a full stop and a space.
The code lexicon might be the binary symbols $\{0,1\}$ that are instantiated by electrical on/off pulses.
Given an encoding scheme, any string of Latin symbols can be converted into a sequence of 0s and 1s, and thus transmitted as a signal across a wire.
The decoder has a duplicate set of Latin symbols from which it must pick out the right symbols in the right order; the signal enables it to perform this task successfully.

As an engineering science, communication theory is concerned with reconstructing symbol strings \textit{efficiently}, which means encoding and transmitting signals with as little cost as possible.
Electrical wires require power and time to operate.
Communication theory can be thought of as a collection of tools and methods enabling an optimal trade-off between signalling effort and the benefits of accurate string reconstruction.
In the central model the most efficient coding schemes are those that use short sequences of 0s and 1s to represent highly probable source strings.
That is because minimising signalling effort means minimising the number of code symbols transmitted, on average.
Pairing probable source strings with short code strings -- short signals -- is the most efficient procedure.
Therefore, in order to devise a good code, you need to know the probabilities of each source string being produced.
Crucially, that is \textit{all} you need to know.
Whether or not source strings also carry natural language meaning is irrelevant to the problem of designing a code.
Shannon stated this clearly, as the next subsection details.

%%%%%%%%%%%%%%%%%%%%%%%%%%%
%%%%%%%%%%%%%%%%%%%%%%%%%%%
%%%%%%%%%%%%%%%%%%%%%%%%%%%

\subsection{Shannon's Warning}\label{subsec:warning}

Communication theory makes heavy use of the term `information'.
Shannon understood the semantic connotations of the term and took care to fend off misinterpretation.
In the introduction to the first of his foundational papers he writes:

\begin{myquote}
The fundamental problem of communication is that of reproducing at one point either exactly or approximately a message selected at another point. Frequently the messages have \emph{meaning}; that is they refer to or are correlated according to some system with certain physical or conceptual entities. These semantic aspects of communication are irrelevant to the engineering problem.
\par\hspace*{\fill}\citet[379]{shannon1948mathematicalc}, emphasis original
\end{myquote}

\noindent Clearly ``message'' in this context refers to a source string.
Shannon warns that the semantic properties of lexical elements do not affect the process of transmitting and reconstructing them.
To see why this is true, note that strings of Latin symbols need not form English-language words, nor words of any other language.
The problem of reconstructing those strings has a distinct mathematical sense, regardless of the strings' natural language implications.

In 1949 Shannon's papers were released in a single volume with prefatory remarks by Warren Weaver \citep{shannon1949mathematical}.
One of Weaver's comments expands on Shannon's earlier technical statement:

\begin{myquote}
In fact, two messages, one of which is heavily loaded with meaning and the other of which is pure nonsense, can be exactly equivalent, from the present viewpoint, as regards information. It is this, undoubtedly, that Shannon means when he says that ``the semantic aspects of communication are irrelevant to the engineering aspects.''
\par\hspace*{\fill}\citet[8]{shannon1949mathematical}
\end{myquote}

\noindent Weaver misquotes Shannon (``\emph{the} semantic aspects'' instead of ``\emph{these} semantic aspects''; ``the engineering \textit{aspects}'' instead of ``the engineering \textit{problem}'').
In context the mistake is insignificant, because the preceding sentences demonstrate that Weaver interprets the point accurately.
By `information' he means a certain property of source strings, defined as the optimal number of symbols in the corresponding code string.
This measure is more commonly known as \textbf{surprisal}, and is defined as $\log{\frac{1}{p(x)}}$ for a message $x$ that is produced at the source with probability $p(x)$.
It is clear that two messages -- two source strings -- can have the same surprisal, with one being a meaningful sentence of a natural language and the other being nonsense.
One need only define a source that produces a meaningful sentence and a nonsensical sentence with the same probability.

In context the misquote is unproblematic, but out of context Weaver can be read as stating what I will eventually deny: that the semantic properties of both the source string \textit{and the code string} are irrelevant to the well-functioning of engineered communication systems.
To dispel any doubt, I endorse Shannon's original claim.
I take it to be as follows:

\begin{myquote}
{\sc Shannon's Warning}: In the central model, once the statistical properties of source strings have been taken into account, the semantic properties of source strings are irrelevant to the engineering problem of communication.
\end{myquote}

\noindent The meanings of source strings are not represented in the mathematics of communication theory.
{\sc Shannon's Warning} tells us that finding an efficient solution to the fundamental problem of communication requires knowing only the statistical properties of source strings, not their meanings.

Philosophers took heed of {\sc Shannon's Warning}.
Over the years, however, it mutated into a different claim.


%%%%%%%%%%%%%%%%%%%%%%%%%
%%%%%%%%%%%%%%%%%%%%%%%%%
%%%%%%%%%%%%%%%%%%%%%%%%%
\subsection{Philosophers' interpretations of the warning}\label{subsec:warningPhil}

Soon after Shannon's initial publications, \citet{bar-hillel1953semantic} set the standard for philosophical interpretation of communication theory:

\begin{myquote}
The Mathematical Theory of Communication, often referred to also as Theory (of Transmission) of Information, as practised nowadays, is not interested in the content of the symbols whose information it measures. The measures, as defined, for instance, by Shannon, have nothing to do with what these symbols symbolise, but only with the frequency of their occurrence.
\par\hspace*{\fill}\citet[147]{bar-hillel1953semantic}
\end{myquote}

\noindent Like Weaver, Bar-Hillel and Carnap likely understood {\sc Shannon's Warning} correctly; like Weaver, the words they used could be misconstrued.
By referring only to `symbols' they risked conflating source symbols and code symbols.
Nonetheless, the job of philosophers, as Bar-Hillel and Carnap saw it, was to provide a theory of \textit{semantic information} that would capture the aspect Shannon ignored.
They explicitly distinguish two kinds of theory, implying a distinction between two entities or concepts.

Bar-Hillel and Carnap's exposition had significant influence.
\citet[p. 241, n.
1]{dretske1981knowledge} cited them as the best-known sceptics about the relevance of communication theory for philosophical questions about content.
Dretske also offered an interpretation of {\sc Shannon's Warning}:

\begin{myquote}
Communication theory does not tell us what information is.
It ignores questions having to do with the \emph{content} of signals, what \emph{specific information} they carry, in order to describe \emph{how much} information they carry.
In this sense Shannon is surely right: the semantic aspects are irrelevant to the engineering problems.
\par\hspace*{\fill}\citet[41]{dretske1981knowledge}, emphasis original
\end{myquote}

\noindent Two things are worth noticing.
First, Dretske is talking about the content of \textit{signals}, whereas {\sc Shannon's Warning} concerns the semantic properties of source strings.
Second, Dretske repeats Weaver's misquote of Shannon: ``\textit{the} semantic aspects'' instead of ``\textit{these} semantic aspects'' (strictly speaking, Dretske does not use quotation marks -- but earlier on the same page he repeats the misquote along with an endnote reference to Shannon's original statement).
Influenced by Dretske, \citet{dennett1983intentional} repeated Bar-Hillel and Carnap's call for a distinction between mathematical and semantic information:

\begin{myquote}
A more or less standard way of introducing the still imperfectly understood distinction between these two concepts of information is to say that Shannon-Weaver theory measures the \emph{capacity} of information-transmission and information-storage vehicles, but is mute about the \emph{contents} of those channels and vehicles, which will be the topic of the still-to-be-formulated theory of semantic information.
\par\hspace*{\fill}\citet[344]{dennett1983intentional}, emphasis original
\end{myquote}

\noindent Dennett speaks of ``channels and vehicles'', which would presumably include signals.
Like Dretske, the version of the warning operative here is different from the original statement.
\citet[$\S$6]{dennett2017bacteria} is still pursuing this line, and it is nowadays standard to distinguish between two senses of the term `information' in scientific applications.
The Stanford encyclopedia entry `Biological Information' is organised around the distinction, using the labels ``Shannon's concept of information'' and ``Teleosemantic and other richer concepts'' \citep{godfrey-smith2016biological}.
\citet[21]{piccinini2011information} say ``Shannon information does not capture, nor is it intended to capture, the semantic content, or meaning, of signals,'' again focusing on signals rather than source strings.
It is accepted practice to refer to Shannon's formal tools as unrelated to semantic content without further argument: ``I will interpret ‘information’ as ‘semantic information’ (i.e. semantic content), not as Shannon information'' \citep[p. 12 n. 14]{artiga2020signals}; see also \citet[6]{cao2020new} and \citet[1]{kolchinsky2018semantic}.
% ``Shannon offers no analysis of the relation in virtue of which a sign carries information \textit{about} a state of affairs (his interest was in other issues)''  \citep[p. 7, emphasis original]{neander2017mark}; ``One of the most cited quotes by Shannon is that referred to the independence of his theory with respect to semantic issues [...] Shannon’s theory, taken in itself, is purely quantitative: it ignores any issue related to informational content'' \citep[1988-9]{lombardi2015shannon}; see also \citet[6]{cao2020new} and \citet[1]{kolchinsky2018semantic}.

It could plausibly be argued that the distinction these scholars are aiming for is between semantic content and correlational measures like mutual information.
The authors cited above could be read as claiming that signals can bear mutual information without possessing semantic content.
I discuss such claims in section \ref{sec:agnostic}; my point here is just that these writers cannot appeal to Shannon and Weaver to justify their position.
Weaver's quote above uses `information' as synonymous with surprisal, not mutual information; both he and Shannon were referring to a property of source strings, not a correlation between signals and signifieds.
To my knowledge, no explicit argument has been offered that moves from the premise embodied by {\sc Shannon's Warning} to the conclusion that there exists a property of signals, invoked in communication theory, that is distinct from the concept sought by a philosophical theory of content.
Most likely there is no valid argument of this kind.
Dretske misinterpreted the ambiguous comments of Bar-Hillel and Carnap, and as a result philosophers have conflated the true claim that source strings need not have semantic content with the false claim that signals need not have semantic content.

Perhaps the strongest argument for my position is that the mutated form of {\sc Shannon's Warning} is false by the lights of communication theorists themselves.
Signals in the central model have semantic content, and their content is directly relevant to the engineering problem of communication, as we shall now see.


%%%%%%%%%%%%%%%%%%%%
%%%%%%%%%%%%%%%%%%%%
%%%%%%%%%%%%%%%%%%%%
%%%%%%%%%%%%%%%%%%%%
%%%%%%%%%%%%%%%%%%%%
\subsection{The content of a signal in the central model is a source string}\label{subsec:signalContent}

Contemporary discussion of the relevance of communication theory for semantic content focuses on signals.
But nothing about signals can be concluded directly from {\sc Shannon's Warning}, which concerns source strings only.
The mathematical tools of communication theory are indeed blind to the meaning of source strings, but interpretations of the theory require that signals are meaningful: the content of a signal is the identity of a source string.
(For simplicity I assume a one-to-one mapping between source strings and code strings; my substantive points are not affected by loosening this constraint.)
We've already seen quotes from communication theorists attributing content to signals, and I've offered teleosemantics as a theory that cashes out this usage.
Here I will give more context to those quotes, to strengthen the plausibility of my interpretation.

Describing Morse code, \citet[385]{shannon1948mathematicalc} says that letters are ``represented by'' sequences of dots and dashes.
In their widely cited textbook \citet[105]{cover2006elements} say the same.
Shannon further uses the locution ``represented by'' in this context on pages 402 and 405; Cover \& Thomas use this and related terms (including calling encodings ``representations'') on pages 5-6, 130, 134, 218-9, 221 and 301.
Obviously the authors cannot be interpreted as taking a stance on philosophical theories of content.
Nevertheless, their usage is evidence of a pretheoretic notion of representation at work, a relation of signification linking signals with source strings.

Furthermore, standard interpretations of the theorems of communication theory require that signals signify source strings.
Consider the \textbf{first theorem of communication theory}, often called the source coding theorem \citep[$\S$5]{cover2006elements} \citep[$\S$4]{mackay2003information}.
The theorem gives a lower bound on the average number of binary code symbols required to encode source strings of a specified length.
The fewest code symbols required is a function of the \textbf{entropy} of the source, which itself is a function of the probabilities of source symbols.
The theorem clearly embodies a notion of signification or reference: both the question that prompted the theorem (how many code symbols are required?) and the result it offers (the entropy of the source) assume that the code symbols are being used to record the outputs of the source (i.e. the symbols of the source string).
Cover \& Thomas make this clear:

\begin{myquote}
This theorem provides another justification for the definition of entropy rate -- it is the expected number of bits [i.e. code symbols] per [source] symbol required to describe the [source] process.
\par\hspace*{\fill}\citet[115]{cover2006elements}
\end{myquote}

\begin{myquote}
We can design source codes for the most efficient representation of the data. [...] The common representation for all kinds of data uses a binary alphabet. Most modern communication systems are digital, and data are reduced to a binary representation for transmission over the common channel.
\par\hspace*{\fill}\citet[218]{cover2006elements}
\end{myquote}

\noindent The extent to which Cover \& Thomas's use of terms like ``describe'' and ``represent'' corresponds with philosophers' notions of semantic content has not to my knowledge been asked.
Their usage is evidence that some notion of signification is required to make sense of the theorem.
A similar sentiment is found in MacKay's textbook, wherein he models a scientist's experimental setup as a source and the results of the experiment as source outcomes:

\begin{myquote}
We now discuss the information content [entropy] of a source by considering how many bits [code symbols] are needed to describe the outcome of an experiment.
\par\hspace*{\fill}\citet[73]{mackay2003information}
\end{myquote}

\noindent Again, the sense of the entropy measure is intimately bound up with representation, in this case in describing the outcome of the experiment.
Similarly, in a discussion of a special case of the theorem, \citet[397]{shannon1948mathematicalc} speaks of ``the number of bits [code symbols] required to specify the sequence'' of outcomes of the source process.

Overall, communication theorists from Shannon to contemporary textbook authors make use of a notion of semantic content, as indicated by their use of words like `represent', `describe' and `specify'.
There is an intuitive sense to this usage.
Signals in the central model must bear some exploitable relation to source strings, because the decoder is using the signals to reconstruct the original strings.
The specific mapping from signal to source -- the `semantics' of the signalling system -- is determined by the encoding scheme.

Teleosemantics puts theoretical legs under this intuition.
As we saw in section \ref{sec:positive}, the central model is a special case of the basic teleosemantic model.
In order to explain how a receiver can successfully reconstruct a source string, we must appeal to a relation between the signal and the source string.
This relation is the signal's representational content, and it is defined by the encoding scheme.

It might be objected that this is a very thin notion of `representation'.
Surely philosophers who aim to give a naturalist account of representation are looking for something much richer than the relationship between simple electronic signals and their signifieds?
Indeed, the liberality debate constitutes an ongoing discussion about this very question \citep{artiga2016liberal,artiga2022strong,desouzafilho2022dual}.
Teleosemantics is particularly susceptible to the charge of being too liberal in its attribution of representational status, because the conditions implied by figure \ref{fig:teleo} are very easy to satisfy.

To respond, note that my argument is not intended to defend teleosemantics against the charge of liberality.
I am only trying to demonstrate that teleosemantics attributes content to central model signals, and that this accounts for the locutions of actual communication theorists.
Detractors may treat this as a mark against teleosemantics; so be it.
My point is just that the sceptic cannot appeal to {\sc Shannon's Warning} to make their case, because Shannon was talking about source strings rather than signals.
The sceptic needs independent grounds on which to reject the claim that central model signals have content.
Of course, they would then have to account for the locutions of communication theorists themselves.

% To sum up, by dint of joint design, encoder and decoder have a shared proper function to reconstruct the source string at the target.
% They do this by means of an intermediary -- the code string as signal.
% Teleosemantics identifies the relation between signal and source string as the basic form of semantic content.
% This cashes out the pretheoretic usage of terms like `represent', `describe' and `specify' used by communication theorists.

\subsection{A sceptical riposte: symbol manipulation does not bestow content}

Encoding, the sceptic will notice, is the transformation of symbols from one lexicon into another.
Since I am claiming it is the encoding scheme that confers content, my position appears to entail that any process by which symbols of one lexicon are converted into symbols of another confers content.
That is a big problem: it is implausible that manipulating symbols from lexicon $L_1$ into lexicon $L_2$ bestows the symbols of $L_2$ with the content `symbol such-and-such from $L_1$'.
Symbol manipulation is a matter of syntax, not semantics.
If that is all that is happening in an encoding scheme, then it is implausible that central model signals really do have the contents I ascribe.

To respond, I begin by accepting the premise:

\begin{myquote}
\smi{}: Converting symbols of lexicon $L_1$ into symbols of lexicon $L_2$ does not bestow the symbols of $L_2$ with content.
\end{myquote}

\noindent I think we can all agree on \smi{}.
Teleosemantics is liberal, but not that liberal.
The question is whether symbol manipulation is all there is to encoding.
While the term `encoding' might be used in different ways in different branches of science and philosophy, including in ways that imply only symbol manipulation, I contend that its use in communication theory implies something stronger.
Encoded strings are produced as part of a sender-receiver system, in order to be decoded during performance of a joint function.
That makes a difference, because it ensures the system fits the teleosemantic template.
Shuffling symbols does not bestow content, but joint design of sender and receiver does.

Furthermore, although I have focused on symbols in the exposition so far, it turns out that source outcomes need not be symbols at all.
They could be dance steps, military manoeuvres, restaurant locations; anything over which a probability distribution can be defined.
It also turns out that the actions of the receiver need not be exact duplicates of the outcomes at the source; they need only be actions that, combined with source outcomes, yield a cost (via a cost function) for the system as a whole.
If this sounds like the sender-receiver framework associated with \citet{skyrms2010signals} and \citet{lewis1969convention} that's because it is formally equivalent to it \citep{martinez2019deception}.
Sender-receiver games (figure \ref{fig:sr}) are also a special case of the basic teleosemantic model, and the central model just is a sender-receiver game.
Source and target -- state and act -- need not be symbols.
They are just commonly described as such because that is part of the typical use case of communication theory.
The mathematics does not demand that signals be about symbols from a lexicon.
They can be about anything at all.

\begin{myfig}
    {img_srBasic} % filename
    {0.9} % width
    {fig:sr} % label
    {\textbf{A signalling game}.
    In cooperative settings, \textbf{Sender} and \textbf{Receiver} must collaborate to achieve a shared payoff.
    The payoff depends on the receiver's \textbf{Act} and the \textbf{State} observed by the sender.
    Since only the sender has access to the state, it must guide the receiver with a \textbf{Signal}.
    Here, the state plays both roles that teleosemantics distinguishes as proximate and distal states (see figure \ref{fig:teleo}): it is a proximate state because the sender conditions its choice of signal on it, and it is a distal state because the value of the target (the payoff) depends on it.
    Signalling games include payoff matrices that yield reward values from combinations of states and acts; these could instead be represented by a downstream Success? variable as in figure \ref{fig:teleo}.
    Therefore, teleosemantics says that the semantic content of a signal is the identity of the corresponding state.
    See also \citet{martinez2019deception} for the close formal links between communication theory and signalling games.
    } % caption
\end{myfig}

To summarise the entire section, arguments denying the relevance of communication theory for theories of content based on {\sc Shannon's Warning} do not hold water.
A stronger argument adverts to the breadth of application of another of Shannon's mathematical measures: mutual information.
It is to this point I now turn.

%%%%
%%%%
%%%%

\section{Second sceptical argument: Agnostic information}\label{sec:agnostic}

The second route to scepticism about the relevance of communication theory for representational content begins with the claim that mathematical measures defined within communication theory cannot distinguish between representations and non-representations.
In other words, information is agnostic to representational status.
In this section I argue that although this is true of certain mathematical functions like mutual information, communication theory has many more tools at its disposal.
Communication theory does distinguish between signals and non-signals -- it must do in order for its theorems to have sense.
The theory is fundamentally about the costs and benefits of representation and how to trade them off judiciously.
Philosophers undersell the resources available to communication theory by focusing solely on a small set of measures.

\subsection{How scientists use information theory}\label{subsec:scientists}

Soon after Shannon's original text, scientists began to notice that his mathematical tools were of use beyond the context of communications engineering.
Perhaps the most well-known informational measure is \textbf{mutual information}, typically interpreted as the strength of correlation between two variables.
Mutual information has been employed in a diverse range of sciences, including:

\begin{itemize}
    \item Behavioural ecology, to measure the correlation between the honeybee waggle dance and the location of food sources \citep{haldane1954statistical}
    \item Cosmology, to measure the correlation between galaxies' internal morphology and their local environments \citep{pandey2017how}
    \item Evolutionary biology, to show that the correlation between an environmental cue and a fitness-relevant state of affairs is an upper bound on the increased growth rate of an organism conditioning its behaviour on the cue \citep{donaldson-matasci2010fitness}
    \item Linguistics, to measure the co-occurrence of words in a corpus \citep[$\S$4]{hunston2002corpora}
    \item Molecular biology, to measure the correlation between inputs and outputs of a quorum-sensing bacterium \citep{mehta2009information}
    \item Neuroscience, to measure the correlation between neural firings and environmental states \citep[][and references therein]{rathkopf2017neural}
\end{itemize}

\noindent This multifaceted role for mutual information accompanied the emergence of the term \textbf{information theory} to describe Shannon's mathematical tools and their more general application across the sciences.
Today, information theory comprises a set of concepts and measures common to many mathematical and scientific disciplines (see figure \ref{fig:info_theory}).
For better or worse, it has become customary for philosophers to use the term `information theory' to describe both these more general applications and the original context of communications engineering in which they were developed.
Since the claims I make below depend on this distinction being upheld, I will continue to use `information theory' and `communication theory' non-synonymously.

\begin{myfig}
    {img_info_theory} % filename
    {0.9} % width
    {fig:info_theory} % label
    {The relationship between communication theory, information theory, and other mathematical and scientific endeavours.
    From \citet[p. 2, fig. 1.1]{cover2006elements}.
    } % caption
\end{myfig}

The breadth of application of mutual information is at the heart of a second source of scepticism about the relevance of communication theory for theories of content.
Sceptics move from a premise about mutual information to a conclusion about information theory as a whole -- which is then seen as encompassing communication theory.
In the next two subsections I argue that the premise is true but the conclusion is false; no valid argument can link them.

%%%%%%%%%%%%%%%%%%
%%%%%%%%%%%%%%%%%%
%%%%%%%%%%%%%%%%%%
%%%%%%%%%%%%%%%%%%
%%%%%%%%%%%%%%%%%%
\subsection{Mutual information cannot distinguish signals and cues}

Let us begin with a premise that everyone should accept: mutual information cannot distinguish between \textbf{signals} and \textbf{cues}.

Originating in behavioural ecology, the signal/cue distinction highlights the fact that some informational vehicles have the function to provide the information they do, whereas some are `accidentally' informational, used opportunistically by their receivers \citep[$\S$1.2]{maynardsmith2003animal}.
Vehicles selected to serve a communicative role are called signals, while vehicles that fortuitously provide information are called cues.
(In philosophy the term `natural sign' is sometimes used; I am here using `cue' to cover all cases described by that term.)
The waggle dance is a signal because it evolved in the honeybee lineage to serve as an informational vehicle that enables workers to enjoy greater success at foraging or nest-finding \citep{gould1975honey,riley2005flight}.
In contrast, bees' use of the position of the sun in the sky to navigate is a cue, because the sun's location is not an outcome of a process of selection that jointly produced both it and the bees' navigational behaviour.

Mutual information quantifies the strength of a correlation no matter whether its vehicles are signals or cues.
The sheer variety of scientific contexts employing mutual information emphasises this point.
While the correlation between the waggle dance and food locations is due to the fact that the waggle dance is a signal, the correlation between galaxies' morphology and their local environment clearly is not.
The vehicle in the evolutionary model of \citet{donaldson-matasci2010fitness} is definitionally a cue.
The co-occurrence of words in a corpus is not a signal (though the words themselves are representations, or at least combine to produce representations).
Without further detail, it is not clear whether the output of a quorum-sensing bacterium counts as a signal of its input; nonetheless, mutual information between the two can be measured.
Neural firings are sometimes claimed to be representations, but simply measuring the correlation between them and environmental states is not sufficient to establish this \citep{rathkopf2017neural}.

From the mere fact that two things bear a correlational relationship, no conclusion can be drawn about whether one is a signal of the other.
I fully agree with this premise and suggest encapsulating it as follows:

\begin{myquote}
\ami: Mutual information cannot distinguish signals and cues.
\end{myquote}

\noindent The path to scepticism I want to explore is the move from \ami{} to the following claim:

\begin{myquote}
\act: Communication theory cannot distinguish signals and cues.
\end{myquote}

\noindent The latter claim would certainly challenge the relevance of communication theory for theories of representational content.
If communication theory cannot distinguish signals and cues, then there is little hope of it distinguishing more sophisticated kinds of representation like those invoked in cognitive science.
And if it cannot distinguish those, it probably cannot say anything philosophically interesting about them.

I am going to argue that \act{} is false.
Communication theory can and does distinguish signals and cues.
If space permitted I would attempt to reconstruct the arguments of philosophers who appear to move from the true premise to the false conclusion.
Space does not permit, so I will instead note that several writers appear to trade on an ambiguity in the term `Shannon information', treating it as both a synonym for mutual information and a catch-all that exhausts the conceptual repertoire of communication theory \citep[p. 1 and $\S$2]{godfrey-smith2016biological}, \citep[759]{owren2010redefining}, \citep[106]{dennett2017bacteria}, \citep[p. 12, n. 11 and p. 78, n. 5]{shea2018representation}.
The same point has been made by Mart\'{i}nez, using the term `information theory' to cover communication theory too:

\begin{myquote}
Philosophy has understood information theory as a mostly definitional effort: for all philosophers have typically cared, the theory begins and ends with a presentation of what it takes for one random variable (or the worldly feature it models) to carry information about another.
\par\hspace*{\fill}\citet[1216]{martinez2019representations}
\end{myquote}

\noindent There is no valid argument linking the premise \ami{} to the conclusion \act{}, because that conclusion is false.

%%%%%%
%%%%%%
%%%%%%
%%%%%%
\subsection{\act{} is false}\label{subsec:actFalse}

Whereas information theory is a collection of mathematical tools with wide application across the sciences, communication theory is an engineering discipline with the specific goal of designing efficient signalling techniques.
The vehicles transmitted in the central model are signals and the main theorems of communication theory apply to signals.
Consider, on the one hand, the \textbf{second theorem} and \textbf{third theorem} of communication theory, which require that the vehicles they address be signals, and on the other hand \textbf{Kelly's theorem}, which requires only that the vehicle be a cue.
We shall introduce them in turn.

The second theorem of communication theory (sometimes called the noisy channel theorem or the channel coding theorem) determines how accurately a receiver is able to reconstruct a source string given a signal that has been corrupted by noise.
Better encodings combat noise by building redundancy into the signal, enabling the receiver to more accurately reconstruct the source.
The theorem answers a question about receiver performance by attending to the sender's design of the vehicle: different encoding schemes would yield different performance levels.
The vehicle mentioned in this theorem is definitionally a signal; the theorem does not apply to cues.

The third theorem of communication theory (also known as the rate-distortion theorem) addresses a similar question, this time with the added benefit that the receiver need not achieve perfect performance.
Suppose for example that the receiver only needs to correctly reconstruct four out of every five outcomes produced by the source.
The third theorem states that it is possible to determine the minimum transmission rate that the sender must ensure in order for the receiver to perform at the specified level.
(This `rate-distortion' trade-off is another name for the resource-success trade-off discussed earlier.)
The third theorem assumes that the transmission rate is tunable by the sender's choice of encoding scheme.
By invoking a vehicle whose form can be adapted to performance specifications, both the second and third theorems employ a concept of signal rather than cue.

Kelly's theorem, by contrast, concerns the performance of a receiver conditioning its behaviour on a vehicle that bears a fixed level of mutual information with a success-relevant distal state \citep{kelly1956new}.
The theorem states that mutual information is an upper bound on the performance improvement a receiver can enjoy by using the vehicle rather than not using it.
Because the emphasis is on receiver behaviour, and the mutual information is held fixed, the vehicle is conceptualised as a cue.
The theorem has been applied in evolutionary biology \citep{donaldson-matasci2010fitness} where again the vehicle in question is treated as an environmental cue rather than a signal.

Of course, nothing prevents us applying Kelly's theorem to signals too.
My claim is not that theorems about cues do not apply to signals, but that theorems about signals do not apply to cues.
There is an asymmetry in the definition of signals and cues, and the two kinds of theorem are asymmetric in a way that reflects that.
Any vehicle can be treated as a cue simply by failing to specify whether or not it was designed for communicative use.
That is what Kelly's theorem does, in the guise of keeping mutual information fixed and asking how the receiver can make use of it.
(In fact Kelly's prose implies that the vehicle in question is a signal; closer inspection reveals the theorem does not require it to be one.)
The second and third theorems by contrast require that their vehicles be signals, because they ensure a level of functional performance that is only available when the sender tunes the vehicle's production to the features of the channel.
By definition, the second and third theorems cannot be applied to cues.

\citet{calcott2020signals} make a similar point with reference to Skyrms's \citeyearpar{skyrms2010signals} definition of information in signals, arguing that signals do not just carry information about their effects but are difference-makers for their effects.
The central model, like all sender-receiver models, is by definition causal: there must be a causal link between sender and receiver in order for there to be a signalling partnership in the first place.
\citet{rathkopf2017neural} makes a similar argument in the case of neurobiology.

In sum, communication theory distinguishes signals from cues both by providing the means to define signals and by employing theorems that require the vehicles in question to be signals.
The fact that communication theory also contains theorems like Kelly's whose vehicles need only be cues serves to sharpen the point.
\act{} is false.

%%%%%%%%%%%
%%%%%%%%%%%
%%%%%%%%%%%
%%%%%%%%%%%
\subsection{A sceptical riposte: channel capacity is agnostic about content}

Alarm bells are ringing in sceptical ears: I just talked about the second theorem, and how the requirement of a causal link between sender and receiver confirms its vehicles are signals.
But the sceptic knows that the second theorem deals with the \textbf{capacity} of the channel.
The capacity is defined only in terms of the signal before noise and the signal after noise; it is oblivious to the source and target strings -- oblivious to what the signal is actually about.
If the second theorem and the channel capacity it defines are agnostic to the content of signals, is not the sceptic justified in asserting that the definition of semantic content must be found elsewhere than communication theory?
\citet[344]{dennett1983intentional} seems to be making this point when he distinguishes between the ``\textit{capacity} of information-transmission and information-storage vehicles'' and their contents, stating that Shannon's theory deals only with capacity.
More recently he reasserts the claim:

\begin{myquote}
Shannon devised a way of \textit{measuring} information, independently of what the information was \textit{about}, rather like measuring \textit{volume} of liquid, independently of which liquid was being measured. (Imagine someone bragging about owning lots of quarts and gallons and not having an answer when asked, ``Quarts of what -- paint, wine, milk, gasoline?'')
\par\hspace*{\fill}\citet[106]{dennett2017bacteria}, emphasis original
\end{myquote}

\noindent The present author has heard similar arguments from philosophers in conversation, moving from a claim about channel capacity to the claim that communication theory cannot specify the contents of signals.

Before responding, I accept that one of the premises of the sceptical riposte is true:

\begin{myquote}
\cia{}: Channel capacity can be measured without specifying the contents of signals transmitted through the channel.
\end{myquote}

\noindent Channels, characterised by the conditional distribution of signal-after-noise given signal-before-noise, are general-purpose in that the outcomes of any source can be encoded by the code symbols.
Storage devices like hard drives and transmission media like fibre-optic cables can be assigned measures of capacity without regard to what they are storing or transmitting.
Any channel can in principle be used to communicate anything.
The honeybee waggle dance could be used to communicate military instructions if the field commander sending the message had sufficiently fine-grained control over the placement of food sources in the bees' locale.
Measuring the capacity of the dance cannot tell you what, on a given occasion, is being communicated by it.

However, the general-purpose nature of channels does not entail that communication theory cannot attribute content to signals.
Signals in the central model (for example) do have content as soon as a source and an encoding scheme are specified: the contents of signals are the different outcomes of the source.
If an army managed to employ the waggle dance to transmit military instructions, each dance would gain a specific instruction as one of its semantic contents in accordance with the code devised ahead of time by the human communicators.
Communication theory and teleosemantics both deliver this result.
That a channel can be used to transmit anything does not mean communication theory remains forever agnostic about what is actually being transmitted in a given circumstance.
In terms of Dennett's analogy, the theory both measures the volume of a liquid and tells you what liquid is being measured.
Dennett really does appear to be saying that Shannon's formal work encompasses only measures like mutual information and channel capacity.
If he means to say this, he is wrong.

To summarise the entire section, true premises about the broad application of mutual information in science and the general-purpose nature of channels provide no reason to be sceptical about the relevance of communication theory for theories of content.




\printbibliography

\end{document}
