\begin{abstract}
\noindent Prominent views about cognitive representations share a premise: that the only relevance communication theory could have for representational content is via measures of correlation. Here I challenge that premise by rejecting two common misconceptions: that Claude Shannon said that the meanings of signals are irrelevant for communication theory (he didn't and they aren't), and that since correlational measures can't distinguish representations from natural signs, communication theory can't distinguish them either (the premise is true but the conclusion is false and the argument is invalid).
\end{abstract}