\begin{myfig}
    {img_srBasic} % filename
    {0.9} % width
    {fig:sr} % label
    {\textbf{A signalling game}.
    In cooperative settings, \textbf{Sender} and \textbf{Receiver} must collaborate to achieve a shared payoff.
    The payoff depends on the receiver's \textbf{Act} and the \textbf{State} observed by the sender.
    Since only the sender has access to the state, it must guide the receiver with a \textbf{Signal}.
    Here, the state plays both roles that teleosemantics distinguishes as proximate and distal states (see figure \ref{fig:teleo}): it is a proximate state because the sender conditions its choice of signal on it, and it is a distal state because the value of the target (the payoff) depends on it.
    Signalling games include payoff matrices that yield reward values from combinations of states and acts; these could instead be represented by a downstream Success? variable as in figure \ref{fig:teleo}.
    Therefore, teleosemantics says that the semantic content of a signal is the identity of the corresponding state.
    See also \citet{martinez2019deception} for the close formal links between communication theory and signalling games.
    } % caption
\end{myfig}