\begin{myfig}
    {img_srBasic} % filename
    {0.9} % width
    {fig:sr} % label
    {A signalling game.
    In cooperative settings, \textbf{Sender} and \textbf{Receiver} must collaborate to achieve a shared payoff at the \textbf{Target}.
    Whether the receiver obtains a payoff depends on its own act (node not shown) and the value of \textbf{State}.
    Since only the sender has access to the state, it must guide the receiver with a \textbf{Signal}.
    Here, the State plays both roles that teleosemantics distinguishes as proximate and distal States (see figure \ref{fig:teleo}): it is a proximate state because the Sender conditions its choice of signal on it, and it is a distal state because the value of the target (the payoff) depends on it.
    See also \citet{martinez2019deception} for the close formal links between communication theory and signalling games.
    } % caption
\end{myfig}